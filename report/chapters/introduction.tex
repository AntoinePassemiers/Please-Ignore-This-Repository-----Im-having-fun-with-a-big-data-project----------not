\chapter{Introduction}

The objective of this assignment is to implement a scalable online algorithm
that is able to perform forecasting in a distributed fashion.
More specifically, such algorithm should be able to predict the values of
some explained variables based on the values of some explanatory variables
by having recourse to linear models. In the framework of this project,
the Recursive Least Squares (RLS) algorithm has been privileged as it is
capable of learning the coefficients of a linear model incrementally.

The implementation extends the notebooks that have been provided
along with the project statement. There are mainly three extensions of these
notebooks:
\begin{itemize}
    \item The coefficients of the underlying linear models are generated randomly
        using a multivariate Gaussian distribution. The parameters of the later
        are drawn at random using a normal-Wishart distribution and fixed
        before the learning process starts.
        Also, multiple output/explained variables are returned instead of a single
        one.
    \item The system is made scalable with respect to the number of models to be
        run in parallel. Running multiple models in parallel with different
        hyper-parameters allows for fast validation and computation of the
        optimal hyper-parameter values. In this project, the only hyper-parameter
        to be tuned is the forgetting factor of the RLS algorithm.
    \item The system is made scalable with respect to the number of explained
        variables. When the number of output variables is low, the RLS update
        function relies on NumPy vectorization, but is actually run in parallel
        when the number of output variables exceeds a given threshold.
        This design choice allows for maximum flexibility.
\end{itemize}

Data samples are sent by a Kafka producer and retrieved by a Kafka consumer
as a Spark DStream. The later is then update the models' coefficients
in a distributed manner. 
